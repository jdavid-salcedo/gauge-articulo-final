% Created 2023-06-23 Fr 17:00
% Intended LaTeX compiler: lualatex
\documentclass[
                egregdoesnotlikesansseriftitles,
                paper=a4,
                fontsize=13pt,
                DIV=calc]{scrarticle}
\input{articlepreamble.sty}
\usepackage{graphicx}
\usepackage{longtable}
\usepackage{wrapfig}
\usepackage{rotating}
\usepackage[normalem]{ulem}
\usepackage{amsmath}
\usepackage{amssymb}
\usepackage{capt-of}
\usepackage{hyperref}
\usepackage[outputdir=./build]{minted}
\usepackage{svg}
\usepackage{rotfloat}
\author{Juan David Salcedo Hernández}
\date{\today}
\title{Una visión sobre las teorías gauge y cómo especificar nuevos modelos en física de partículas mediante el uso de la herramienta computacional \texttt{minimal\_lagrangians}. Verificaciones y perspectivas}
\hypersetup{
 pdfauthor={Juan David Salcedo Hernández},
 pdftitle={Una visión sobre las teorías gauge y cómo especificar nuevos modelos en física de partículas mediante el uso de la herramienta computacional \texttt{minimal\_lagrangians}. Verificaciones y perspectivas},
 pdfkeywords={},
 pdfsubject={},
 pdfcreator={Emacs 28.2 (Org mode 9.6.1)}, 
 pdflang={English}}
\usepackage{biblatex}
\addbibresource{/home/juan/Personal/latex/2023/proyecto-final/references.bib}
\begin{document}

\maketitle
\tableofcontents


\section{Introducción}
\label{sec:org2d0c3e2}
Referencias a utilizar:
Para variedades diferenciables el libro \autocite{ballmann2015}. Para fibrados y demás
preámbulos matemáticos a las teorías gauge los libros \autocite{hamilton2017},
\autocite{deFaria2010} y \autocite{baum2009}. Para el programa computacional el artículo
\autocite{may2021minimal}.

Una teoría Gauge es un tipo de teoría de campos sobre la cual se exige simetría
(es decir, invarianza) de los campos empleados bajo una cierta familia de
transformaciones, a la cual, a su vez, se le exige conformar un grupo de Lie, a
saber, un grupo bajo la composición que adicionalmente posea la estructura topológica
de una veriedad diferenciable. La mayoría de los grupos de Lie que son de
relevancia en la física son, de hecho, subgrupos del grupo lineal general
\(GL(V)\) de todos los operadores lineales sobre un \(\mathbb{C}\)-espacio vectorial
de dimensión finita \(V\). Dado que se pueden construir representaciones (es
decir, homomorfismos) que mapean a los \(GL(V)\) en los grupos de matrices invertibles
\(GL(n)\) mediante la selección de una base de \(V\), son estos últimos los que mayor
importancia cobran en el estudio de las teorías físicas.

Desde el punto de vista abstracto, el Lagrangiano correspondiente a una teoría
de campos que involucre diferentes campos \(\varphi_1, \ldots, \varphi_n\) sobre
el espacio-tiempo de Minkowski \(M = \mathbb{R}^{1,3}\) con métrica plana \(\eta =
\text{diag}\,(-1, 1, 1, 1)\) no es más que una función \(\mathcal{L}\) de todos
los campos, la cual contiene toda la dinámica y las interacciones de y entre dichos
campos. Si el Lagrangiano de las teorías clásicas de campos daba lugar a las
ecuaciones de movimiento a través de la minimización de la acción, en la teoría
cuántica de campos cobrará su rol en la formulación de integrales de caminos,
donde se usará para calcular la dinámica de las partículas elementales.

Es así como, dado el contenido de campos de una teoría Gauge específica y el
grupo de simetría bajo el cual tales campos deben ser invariantes, se hace
palpable la necesidad de construir la función Lagrangiana. En principio hay, por
supuesto, una cantidad incontable de Lagrangianos que se podrían construir dado
el conjunto de campos; sin embargo, aquellos que resultan tener alguna
significancia en la física son restringidos por la existencia de simetrías, la
imposición de la condición de renormalización y la ausencia de anomalías Gauge.

Es lícito pensar en la implementación de tales condiciones sobre el Lagrangiano
para ser capaces de, dados los campos que tienen algún rol en la teoría,
escribir el Lagrangiano más general posible. Con tal fin, se introducen
herramientas computacionales orientadas a la construcción genérica de
lagrangianos en teorías Gauge en física de partículas. En este informe
explicamos, en específico, el funcionamiento del programa \texttt{minimal\_lagrangians}
(Simon may) y se adelantan pruebas de calidad sobre el output que produce para
el bien conocido modelo estándar de partículas elementales, así como para el
modelo escotogénico (ref). Adicionalmente, se demuestra su correcto
funcionamiento para la generación de los archivos que especifican el contenido
de una teoría Gauge en SARAH.

\subsection{Simetría del Lagrangiano}
\label{sec:orgdbcfdb1}
¿Qué significa? ¿Cómo restringe el Lagrangiano?

\subsection{Renormalizablilidad del Lagrangiano}
\label{sec:org2dbd824}
¿Qué significa? ¿Cómo restringe el Lagrangiano?

\subsection{Ausencia de anomalías Gauge en el Lagrangiano}
\label{sec:org86e9f32}
¿Qué significa? ¿Cómo restringe el Lagrangiano?

\section{Contexto matemático sobre las teorías Gauge: una revisión compacta}
\label{sec:orgc3ac83c}

\subsection{Motivación}
\label{sec:org0828073}
Como trataremos de ilustrar a continuación, las teorías de gauge se ocupan
fundamentalmente de \textbf{fibrados principales} y \textbf{fibrados vectoriales asociados}.
El conceptos matemático subyacente a estos dos objetos es el \textbf{haz fibrado}, que
puede ser pensado como un producto directo no trivial (``retorcido'') entre una
variedad (base) \(M\) y una variedad (fibrada) \(F\). Los fibrados prinicpales y
vectoriales serán entonces haces fibrados cuyas fibras son grupos de Lie y
espacios vectoriales topológicos, respectivamente. En estos casos tendremos
estructuras especiales sobre el fibrado.

Adicionalmente, el concepto de \textbf{conexión sobre un fibrado principal} será de
grandísima importancia para describir los \textbf{campos gauge}, cuyas excitaciones
particulares en la teoría cuántica de campos dan lugar a los \textbf{bosones gauge},
que transmiten interacciones. Por su parte, los \textbf{campos de materia} en el modelo
estándar (por ejemplo quarks y leptones) y los \textbf{campos escalares} (como el
Higgs) serán \textbf{secciones de fibrados vectoriales} asociadas a un fibrado
principal (y retorcidas por \textbf{fibrados de espinores} en el caso de los
fermiones). Tendremos, por tanto, las correspondencias siguientes:
\begin{itemize}
\item Los campos gauge serán conexiones en un fibrado principal sobre el espacio de
Minkowski.
\item Los campos de materia y los camppos escalares serán secciones sobre un fibrado
vectorial asociado al fibrado principal.
\end{itemize}

La razón de la interacción entre los campos gauge y los de materia y escalares
es que están ambos relacionados con el mismo fibrado principal.

Por supuesto, nada de esto tiene mucho sentido hasta el momento, pero en lo que
sigue nos ocuparemos de introducir las definiciones necesarias para comprender
la estructura de una teoría gauge como el modelo estándar.

\subsection{Variedades diferenciables}
\label{sec:orgb8aec5c}
\ldots{}

\subsection{Haces fibrados}
\label{sec:org4e569cd}
Sean \(E, M\) variedades suaves y \(\pi : E \longrightarrow M\) una aplicación
diferenciable sobreyectiva entre ellas.

\subsection{Conexiones y curvatura}
\label{sec:org9970201}
\ldots{}

\subsection{Espinores}
\label{sec:org17f7eaa}
\ldots{}

\subsection{Campos gauge y partículas}
\label{sec:org61d3807}
\ldots{}

\section{\texttt{minimal\_lagrangians}: recetas computacionales para la especificación de nuevos campos de materia}
\label{sec:org5d56f62}

Explicar toda la estructura del programa \texttt{minimal\_lagrangians} con detalles\ldots{}

El trasfondo del programa se encuentra en la carpeta \texttt{min\_lag}, donde existen
tres ficheros principales, en orden estructural:
\begin{itemize}
\item \texttt{fields.py}: contiene las clases que definen los diferentes campos de la teoría gauge.
\item \texttt{terms.py}: contiene las clases y métodos que establecen las posibles combinaciones de campos bajo las restricciones de simetría, renormalización y ausencia de anomalías gauge.
\item \texttt{models.py}: \ldots{}
\end{itemize}

\section{Verificación de la salida del programa \texttt{minimal\_lagrangians}}
\label{sec:org23d7686}
\subsection{Modelo estándar}
\label{sec:org1a24f50}
Presentamos el programa que se utiliza para testear minimal lagrangians para el modelo estándar.
\subsection{Modelo escotogénico}
\label{sec:orgc54d82e}
Presentamos el programa que se utiliza para testear minimal lagrangians con el modelo escotogénico.

\section{Perspectivas y conclusiones}
\label{sec:org299f159}

\section{Referencias}
\label{sec:org167ac07}
\printbibliography
\end{document}